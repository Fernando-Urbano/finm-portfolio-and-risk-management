\documentclass{article}

% Language setting
% Replace `english' with e.g. `spanish' to change the document language
\usepackage{graphicx} % Required for inserting images
\usepackage{amsmath}
\usepackage{amsfonts}
\usepackage{amssymb}
\usepackage{listings}
\usepackage{xcolor}
\setlength{\parindent}{0pt}
\setlength{\parskip}{0.5em}
\usepackage[colorlinks=true, linkcolor=blue, urlcolor=blue]{hyperref}
\usepackage{geometry}
\usepackage{titlesec}


% Set page size and margins
% Replace `letterpaper' with`a4paper' for UK/EU standard size


\DeclareMathOperator*{\argmin}{arg\,min}
\DeclareMathOperator*{\argmax}{arg\,max}

% Setup the listings package
\lstdefinestyle{mystyle}{
    backgroundcolor=\color{backcolour},   
    commentstyle=\color{codegreen},
    keywordstyle=\color{magenta},
    numberstyle=\tiny\color{codegray},
    stringstyle=\color{codepurple},
    basicstyle=\ttfamily\footnotesize,
    breakatwhitespace=false,         
    breaklines=true,                 
    captionpos=b,                    
    keepspaces=true,                 
    numbers=left,                    
    numbersep=5pt,                  
    showspaces=false,                
    showstringspaces=false,
    showtabs=false,                  
    tabsize=2
}

\newcommand{\redbold}[1]{\textbf{\textcolor{red}{#1}}}

\lstset{style=mystyle}

\title{Some Takeaways - Tangency Portfolio Factor Model: Derivation of In-Sample \( R^2 = 1 \) in the Cross-Sectional Regression}
\author{Fernando Urbano}
\date{Autumn 2024}

\begin{document}
\maketitle

When we use the tangency portfolio for a given set of assets as the factor model for the same assets, the tangency portfolio explains 100\% of the variability in the cross-section. Meaning that, in-sample, it is the perfect pricing model ($R^2 = 1$).

Why is that the case?

\section{Reminders}

Weights in the tangency portfolio:

\[
\mathbf{w}_p = \frac{\Sigma^{-1} \boldsymbol{\mu}}{\mathbf{1}^\top \Sigma^{-1} \boldsymbol{\mu}}
\]

The denominator, \( \mathbf{1}^\top \Sigma^{-1} \boldsymbol{\mu} \), serves as a scaling factor to normalize the portfolio weights.

The $\beta$ value in a regression with a single feature:

$$
\beta = \dfrac{\text{Cov}(y, x)}{\text{Var}(y)}
$$

In our case:

$$
\beta = \dfrac{\text{Cov}(r_i, r_p)}{\text{Var}(r_p)}
$$

Where:

\begin{itemize}
    \item $r_p$ is the time-series of returns of the tangency portfolio.
    \item $r_p$ is the time-series of returns of the asset $i$.
\end{itemize}

\section{Covariance between $r_i$ and Tangency Portfolio}

The covariance between the return of asset \( i \) and the tangency portfolio \( r_p \) is:
\[
\operatorname{Cov}(r_i, r_p) = \operatorname{Cov}\left(r_i, \mathbf{w}_p^\top \mathbf{r}\right) = \mathbf{w}_p^\top \operatorname{Cov}(r_i, \mathbf{r}) = \mathbf{w}_p^\top \Sigma_{\cdot i}
\]

where $\mathbf{w}_p^\top \Sigma_{\cdot i}$ represents the $i$ column of $\Sigma$ or the $i$ value of the vector $\mathbf{w}_p^\top \Sigma$.

Since:

\[
\Sigma \mathbf{w}_p = \frac{\boldsymbol{\mu}}{\mathbf{1}^\top \Sigma^{-1} \boldsymbol{\mu}},
\]

we have:
\[
\operatorname{Cov}(r_i, r_p) = \frac{\mu_i}{\mathbf{1}^\top \Sigma^{-1} \boldsymbol{\mu}}.
\]

\section{Variance of the Tangency Portfolio}

The variance of the tangency portfolio \( r_p \) is:
\[
\operatorname{Var}(r_p) = \mathbf{w}_p^\top \Sigma \mathbf{w}_p = \left( \frac{\Sigma^{-1} \boldsymbol{\mu}}{\mathbf{1}^\top \Sigma^{-1} \boldsymbol{\mu}} \right)^\top \Sigma \left( \frac{\Sigma^{-1} \boldsymbol{\mu}}{\mathbf{1}^\top \Sigma^{-1} \boldsymbol{\mu}} \right).
\]

Which simplifies to:

\[
\operatorname{Var}(r_p) = \frac{\boldsymbol{\mu}^\top \Sigma^{-1} \boldsymbol{\mu}}{\left( \mathbf{1}^\top \Sigma^{-1} \boldsymbol{\mu} \right)^2}.
\]

\section{Beta of $r_i$ to the Tangency Portfolio}
The beta of asset \( i \) relative to the tangency portfolio is:

\[
\beta_i
= \frac{\operatorname{Cov}(r_i, r_p)}{\operatorname{Var}(r_p)}
= \dfrac{ \dfrac{\mu_i}{\mathbf{1}^\top \Sigma^{-1} \boldsymbol{\mu}} }{ \dfrac{\boldsymbol{\mu}^\top \Sigma^{-1} \boldsymbol{\mu}}{\left( \mathbf{1}^\top \Sigma^{-1} \boldsymbol{\mu} \right)^2} }
\]

Simplifying:
\[
\beta_i = \frac{\mu_i \cdot \mathbf{1}^\top \Sigma^{-1} \boldsymbol{\mu}}{\boldsymbol{\mu}^\top \Sigma^{-1} \boldsymbol{\mu}}.
\]

\section{Expected Returns as a Function of Beta}

Rearranging the expression for \( \beta_i \), we find:

\[
\mu_i = \beta_i \cdot \frac{\boldsymbol{\mu}^\top \Sigma^{-1} \boldsymbol{\mu}}{\mathbf{1}^\top \Sigma^{-1} \boldsymbol{\mu}}.
\]

This shows a perfect linear relationship between \( \mu_i \) and \( \beta_i \), with the slope given by:

\[
\frac{\boldsymbol{\mu}^\top \Sigma^{-1} \boldsymbol{\mu}}{\mathbf{1}^\top \Sigma^{-1} \boldsymbol{\mu}}
\]

Thus, the relationship between the asset and its $\beta$, for any asset $i$, is given by this \textbf{same} slope.

\section{Cross-Sectional Regression}
When performing a cross-sectional regression:

\[
\mu_i = \eta + \lambda \beta_i + \nu_i
\]

From the relationship \( \mu_i = \beta_i \cdot \frac{\boldsymbol{\mu}^\top \Sigma^{-1} \boldsymbol{\mu}}{\mathbf{1}^\top \Sigma^{-1} \boldsymbol{\mu}} \), we find:

\[
\hat{\lambda} = \frac{\boldsymbol{\mu}^\top \Sigma^{-1} \boldsymbol{\mu}}{\mathbf{1}^\top \Sigma^{-1} \boldsymbol{\mu}}
\]

\[
\hat{\eta} = 0
\]

Which connects perfectly all the $\mu_i$, leadind to $\nu_i = 0 \quad \forall \quad i$. Consequently:

\[
R^2 = 1.
\]

Thus, in-sample the tangency portfolio is the perfect factor pricing model.

\end{document}