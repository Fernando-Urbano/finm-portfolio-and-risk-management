\documentclass{article}
\usepackage{amsmath}

\begin{document}

\section*{Derivation of Cross-Sectional $R^2 = 1$ for the Tangency Portfolio}

The tangency portfolio maximizes the Sharpe ratio, and when we use it to form a single-factor model of expected returns, the $R^2$ of the cross-sectional regression of expected returns on betas is 100\%. Here’s the mathematical derivation.

\subsection*{1. Time-Series Regression to Estimate Betas}

For each asset $i$, we perform a time-series regression of its excess returns on the excess returns of the tangency portfolio p:
\begin{equation}
\tilde{r}_{i,t} = \alpha_i + \beta_i \tilde{r}_{p,t} + \epsilon_{i,t}
\end{equation}
where:

\begin{itemize}
\item $\tilde{r}{i,t}$ is the excess return of asset $i$ at time t,
\item $\tilde{r}{p,t}$ is the excess return of the tangency portfolio at time t,
\item $\beta_i$ is the estimated beta of asset $i$ with respect to the tangency portfolio,
\item $\alpha_i$ is the intercept term,
\item $\epsilon_{i,t}$ is the regression residual.
\end{itemize}

\subsection*{2. Analytical Formula for the Tangency Portfolio}

The tangency portfolio weights, which maximize the Sharpe ratio, are given by:

\begin{equation}
\mathbf{w}_p = \frac{\boldsymbol{\Sigma}^{-1} \tilde{\mathbf{r}}}{\mathbf{1}’ \boldsymbol{\Sigma}^{-1} \tilde{\mathbf{r}}}
\end{equation}

where:

\begin{itemize}

\item $\mathbf{w}_p$ is the vector of weights in the tangency portfolio,
\item \( \boldsymbol{\Sigma} \) is the covariance matrix of asset returns,
\item $\tilde{\mathbf{r}}$is the vector of expected excess returns,
\item $\mathbf{1}$ is a vector of ones.
\end{itemize}

Let’s denote the normalization constant by:
\begin{equation}
k = \frac{1}{\mathbf{1}’ \boldsymbol{\Sigma}^{-1} \tilde{\mathbf{r}}}
\end{equation}
Then, we can express the tangency portfolio weights as:
\begin{equation}
\mathbf{w}_p = k \boldsymbol{\Sigma}^{-1} \tilde{\mathbf{r}}
\end{equation}

\subsection*{3. Expected Excess Returns and Covariances}

The covariance between asset $i$ and the tangency portfolio is:
\begin{equation}
\text{Cov}(\tilde{r}_i, \tilde{r}_p) = \mathbf{w}_p’ \boldsymbol{\sigma}_i
\end{equation}
where \( \boldsymbol{\sigma}_i \) is the i -th column of \( \boldsymbol{\Sigma} \).

The variance of the tangency portfolio is:
\begin{equation}
\text{Var}(\tilde{r}_p) = \mathbf{w}_p’ \boldsymbol{\Sigma} \mathbf{w}_p
\end{equation}

\subsection*{4. Calculating Betas}

The beta of asset $i$ with respect to the tangency portfolio is:
\begin{equation}
\beta_i = \frac{\text{Cov}(\tilde{r}_i, \tilde{r}_p)}{\text{Var}(\tilde{r}_p)} = \frac{\mathbf{w}_p’ \boldsymbol{\sigma}_i}{\mathbf{w}_p’ \boldsymbol{\Sigma} \mathbf{w}_p}
\end{equation}

\subsection*{5. Relationship Between Expected Returns and Betas}

Using the tangency portfolio weights, we have:

\begin{equation}
\tilde{\mathbf{r}} = \frac{1}{k} \boldsymbol{\Sigma} \mathbf{w}_p
\end{equation}

For each asset i, we find:

\begin{equation}
E[\tilde{r}_i] = \frac{1}{k} (\boldsymbol{\Sigma} \mathbf{w}_p)_i = \frac{1}{k} \text{Cov}(\tilde{r}_i, \tilde{r}_p)
\end{equation}

Given that $\beta_i = \frac{\text{Cov}(\tilde{r}_i, \tilde{r}_p)}{\text{Var}(\tilde{r}_p)}$, we obtain:

\begin{equation}
E[\tilde{r}_i] = \left( \frac{1}{k} \text{Var}(\tilde{r}_p) \right) \beta_i
\end{equation}

Calculating k using the expected return and variance of the tangency portfolio:
\begin{equation}
E[\tilde{r}_p] = \mathbf{w}_p’ \tilde{\mathbf{r}} = \mathbf{w}_p’ \left( \frac{1}{k} \boldsymbol{\Sigma} \mathbf{w}_p \right) = \frac{1}{k} \text{Var}(\tilde{r}_p)
\end{equation}
Thus,
\begin{equation}
k = \frac{\text{Var}(\tilde{r}_p)}{E[\tilde{r}_p]}
\end{equation}

Substituting back, we find:
\begin{equation}
E[\tilde{r}_i] = E[\tilde{r}_p] \beta_i
\end{equation}

\subsection*{6. Cross-Sectional Regression and $R^2$ Calculation}

The cross-sectional regression equation for expected returns is:
\begin{equation}
E[\tilde{r}_i] = \gamma_0 + \gamma_1 \beta_i + \varepsilon_i
\end{equation}

where:
\begin{itemize}
\item $\gamma_0 = 0$,
\item $\gamma_1 = E[\tilde{r}_p]$,
\item $\varepsilon_i = 0$.
\end{itemize}

Since $\varepsilon_i = 0$ for all i, the sum of squared residuals (SSR) is zero:

\begin{equation}
\text{SSR} = \sum_{i} \varepsilon_i^2 = 0
\end{equation}

The $R^2$ of the regression is then:
\begin{equation}
R^2 = 1 - \frac{\text{SSR}}{\text{SST}} = 1
\end{equation}

\section*{Conclusion}

Thus, by performing the time-series regression to estimate betas and using the analytical formula for the tangency portfolio, we have shown that expected excess returns are perfectly explained by betas in the cross-sectional regression, resulting in an $R^2$ of 1.

\end{document}