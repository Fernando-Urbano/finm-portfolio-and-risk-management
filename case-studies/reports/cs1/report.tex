\documentclass{article}

% Language setting
% Replace `english' with e.g. `spanish' to change the document language
\usepackage{graphicx} % Required for inserting images
\usepackage{amsmath}
\usepackage{listings}
\usepackage{xcolor}
\setlength{\parindent}{0pt}
\setlength{\parskip}{0.5em}
\usepackage[colorlinks=true, linkcolor=blue, urlcolor=blue]{hyperref}
\usepackage{geometry}
\usepackage{titlesec}


% Set page size and margins
% Replace `letterpaper' with`a4paper' for UK/EU standard size


\DeclareMathOperator*{\argmin}{arg\,min}
\DeclareMathOperator*{\argmax}{arg\,max}

% Setup the listings package
\lstdefinestyle{mystyle}{
    backgroundcolor=\color{backcolour},   
    commentstyle=\color{codegreen},
    keywordstyle=\color{magenta},
    numberstyle=\tiny\color{codegray},
    stringstyle=\color{codepurple},
    basicstyle=\ttfamily\footnotesize,
    breakatwhitespace=false,         
    breaklines=true,                 
    captionpos=b,                    
    keepspaces=true,                 
    numbers=left,                    
    numbersep=5pt,                  
    showspaces=false,                
    showstringspaces=false,
    showtabs=false,                  
    tabsize=2
}

\newcommand{\redbold}[1]{\textbf{\textcolor{red}{#1}}}

\lstset{style=mystyle}

\title{Some Takeaways - The Harvard Management Company and Inflation-Protected Bonds}
\author{Fernando Urbano}
\date{Autumn 2024}

\begin{document}
\maketitle

\redbold{Attention: This is not the complete material.} The following are organized notes from the lecture with key takeaways. We highly recommend that you also study the PowerPoint slides and supplement your understanding with your own lecture notes.

\hypertarget{introduction}{%
\section{Introduction}\label{introduction}}

Jack Meyer proposed a significant change to the Harvard Policy
Portfolio, which determined the long-term asset allocation of the
Harvard endowment across various asset classes.

The proposition aimed to drastically reduce the allocation to U.S.
equities and U.S. nominal bonds, while making a substantial investment
(7\% of the portfolio) in U.S. Treasury Inflation-Protected Securities
(TIPS), which had only been added to the portfolio 12 months earlier.

HWC pursued an active strategy to manage its endowment funds. In 1999,
HMC managed 68\% of the endowment assets internally.

Over the past 10 years, this active strategy produced a real
(inflation-adjusted) return of 11.3\% per year after expenses. In
comparison, TIPS returned 2.2\%, U.S. Treasury bonds returned 5.2\%, and
U.S. stocks returned 15.8\% annually.

\hypertarget{policy-portfolio}{%
\subsection{Policy Portfolio}\label{policy-portfolio}}

The Policy Portfolio established the long-term allocation of the
endowment across various asset classes. The management had the
flexibility to make short-term tactical adjustments within predefined
minimum and maximum limits without seeking prior Board approval. It also
served as a benchmark, utilizing well-known market indexes, to evaluate
the performance of the active investment strategies pursued by the
managers.

The Policy Portfolio would only change in response to:
\begin{enumerate}
    \item Changes in the goals or risk tolerance of the university as an institution;
    \item Changes in capital market assumptions;
    \item The emergence of a new asset class in the market.
\end{enumerate}

\hypertarget{long-term-goal}{%
\subsection{Long-Term Goal}\label{long-term-goal}}

Harvard's long-term goal for the endowment was to distribute 4\% to 5\%
of the endowment annually to the university's schools, while preserving
the real value of the endowment and allowing for some real growth.

Additionally, the endowment received an average of 1\% annually from
gifts.

Therefore, to meet its preservation goal, Harvard's endowment required
an average real return of at least 3\% to 4\% per year. To account for
spending growth, the endowment needed 6\% to 7\% in real returns.

\hypertarget{allocation}{%
\subsection{Allocation}\label{allocation}}

The allocation of the Harvard endowment was based on the mean, variance,
and correlation between assets. Historical data and expert assessments
were also taken into account.

Ultimately, the mean-variance analysis was used to determine the optimal
allocation for each asset class, aiming to minimize portfolio return
variance while achieving the expected return.

\hypertarget{is-it-necessary-to-add-tips}{%
\subsection{Is it Necessary to Add
TIPS?}\label{is-it-necessary-to-add-tips}}

To add TIPS as a new asset class to the Policy Portfolio, conditions (2)
or (3) must apply:
\begin{enumerate}
    \item Changes in capital market assumptions; 
    \item The
    appearance of a new asset class in the market.
\end{enumerate}

\hypertarget{tips}{%
\subsection{TIPS}\label{tips}}

TIPS are bonds whose principal and coupons adjust with the general price
level. The structure of TIPS requires the bond's principal and coupon to
change based on the monthly inflation level, as determined by the CPI.

\hypertarget{asset-allocation-with-tips}{%
\subsection{Asset Allocation with
TIPS}\label{asset-allocation-with-tips}}

TIPS offered real yields ranging from 3.2\% to 4.25\%, compared to 3\%
from regular U.S. Treasuries. The team considered TIPS an attractive
asset for a diversified portfolio due to:
\begin{itemize}
    \item Relatively high yields
    \item Inflation protection characteristics
\end{itemize}

The team believed a 4\% yield was a reasonable estimate for TIPS'
expected real return, especially for an institution like Harvard with a
long-term investment horizon.

\hypertarget{mean-variance-optimization-suggestions}{%
\subsection{Mean-Variance Optimization
Suggestions}\label{mean-variance-optimization-suggestions}}

The optimization results indicated that inflation-protected bonds were
an attractive asset to include in the portfolio.

Jack Meyer evaluated the asset allocations of other university
endowments and decided to recommend the new Policy Portfolio.

\hypertarget{discussion-the-harvard-management-company-and-inflation-protected-bonds}{%
\section{Discussion}\label{discussion-the-harvard-management-company-and-inflation-protected-bonds}}

\hypertarget{why-is-harvard-optimizing-the-buckets-instead-of-optimizing-the-entire-portfolio}{%
\subsection{Why Is Harvard Optimizing the Buckets Instead of Optimizing
the Entire
Portfolio?}\label{why-is-harvard-optimizing-the-buckets-instead-of-optimizing-the-entire-portfolio}}

Handling thousands of assets would result in an unstable optimization
due to a covariance matrix with a very small determinant.

Likewise, splitting the portfolio into layers is not optimal if there is
correlation between asset classes. For instance, a U.S. stock manager
might hold tech stocks, and an emerging market (EM) stock manager might
also hold tech stocks, creating high exposure to the same risk. The
two-layer optimization prevents optimal diversification between classes.
This only works if there is no correlation between classes, which is
rarely the case.

Asset classes often exhibit significant correlations, particularly stock
classes. Among the available assets, IEF (U.S. Treasury bonds) was one
of the few with low correlation to other asset classes. In other words,
even if asset classes are labeled differently, they can behave
similarly.

\hypertarget{correlation-changes-and-the-covariance-matrix-problem}{%
\subsection{Correlation Changes and the Covariance Matrix
Problem}\label{correlation-changes-and-the-covariance-matrix-problem}}

Mean-variance optimization assumes its inputs are fixed, but in reality,
correlations between assets vary significantly between in-sample and
out-of-sample data, creating prediction challenges.

When the covariance matrix has a small determinant, the resulting
portfolio weights are often extreme because the optimization treats
returns, covariances, and variances as certainties. This leads to:
\begin{itemize}
    \item Outweighing assets with slightly better average returns.
    \item ``Crazy''
    long positions on one asset and ``crazy'' short positions on another
    when correlations are high.
\end{itemize}

As a result, the optimization ends up magnifying the impact of both
useful and noisy data. Out-of-sample results diverge significantly from
the in-sample optimal solution.

To mitigate the sensitivity issue in the covariance matrix, one can
adopt a Bayesian or machine learning approach, with regularization as a
key solution.

\hypertarget{can-tips-be-its-own-asset-class}{%
\subsection{Can TIPS Be Its Own Asset
Class?}\label{can-tips-be-its-own-asset-class}}

The Sharpe ratio alone is insufficient to decide whether TIPS should be
classified as a separate asset class. Instead, one should consider the
correlation and characteristics of the asset.

In the case of TIPS, increasing the expected return can greatly impact
the mean-variance optimization. If the expected return is perceived to
be higher and significantly improves the Sharpe ratio, it might be
reasonable to consider TIPS as a separate asset class.

\hypertarget{impact-of-dropping-or-changing-tips-returns}{%
\subsubsection{Impact of Dropping or Changing TIPS
Returns}\label{impact-of-dropping-or-changing-tips-returns}}

Removing TIPS from the portfolio has minimal impact on the optimization.
However, changing the expected returns of TIPS (to better reflect future
expectations) can increase their portfolio weight by around threefold.

This increase is due to the covariance sensitivity. A slight improvement
in TIPS returns would increase the portfolio's Sharpe ratio by 10 basis
points, a significant enhancement.

\hypertarget{addressing-the-instability-in-the-covariance-matrix}{%
\subsection{Addressing the Instability in the Covariance
Matrix}\label{addressing-the-instability-in-the-covariance-matrix}}

Regularization is a useful method for addressing instability in the
covariance matrix, especially when: 
\begin{itemize}
    \item Many assets are involved in the
    optimization.
    \item The matrix has high covariances relative to variances
    (high asset correlations).
\end{itemize}

\hypertarget{optimizing-the-sp-500-securities}{%
\subsection{In Class Example: Optimizing the S\&P 500
Securities}\label{optimizing-the-sp-500-securities}}

In our sample, we performed a mean-variance optimization on the 500 S\&P
securities and constructed the following strategies: 
\begin{itemize}
    \item Equally weighted
    \item Parity (volatility-adjusted)
    \item Traditional mean-variance optimization
    \item Non-negative weights
    \item RIDGE regularization
    \item LASSO regularization
\end{itemize}

The equal-weighted and parity portfolios showed a 99.7\% similarity. The
non-negative weights portfolio had an 80.7\% correlation with these,
indicating that despite the more complex optimization, results remain
relatively similar.

Out-of-sample, Sharpe ratios dropped significantly, and equal weights
provided the best risk-adjusted excess returns. With hyperparameter
tuning, RIDGE and LASSO would likely outperform equal weighting. Even
without tuning, they still outperformed traditional mean-variance
optimization.

The mean-variance optimization performed worst out-of-sample due to
in-sample overfitting, a problem magnified with the 500x500 covariance
matrix, which had a far smaller determinant than a simpler 11x11 matrix.

Moreover, the gross leverage of the mean-variance optimizer was 250x,
with extreme long and short positions (e.g., 8000\% long, 10000\%
short), and high turnover that would erode returns through trading
costs.

\hypertarget{how-did-harvard-manage-these-issues-in-its-optimization}{%
\subsection{How Did Harvard Manage These Issues in Its
Optimization?}\label{how-did-harvard-manage-these-issues-in-its-optimization}}

The investment committee set constraints on the weights. For instance,
an asset could not be shorted or exceed a certain threshold.

In Harvard's case, the mean-variance optimization included additional
boundary conditions:

\(h_{i}(w)\hspace{3pt}{\leq}\hspace{3pt}d^{max}\)

\(h_{i}(w)\hspace{3pt}{\geq}\hspace{3pt}d^{min}\)

This optimization included two equality constraints and ``2n''
inequality constraints, making the problem more complex. However, the
numerical optimization remained feasible, as it still operated within a
convex, linear program.

\hypertarget{the-problem-with-setting-bounds}{%
\subsubsection{The Problem with Setting Bounds}\label{the-problem-with-setting-bounds}}

The bounds on weights are largely arbitrary. They introduce additional
inequality constraints (``2n''), complicating the optimization.

When performing the optimization, the ``cost'' of each constraint can be
assessed. This helps gauge how far the solution is from the ``optimal''
due to these a priori or institutional constraints.

To avoid arbitrary constraints, regularization approaches have gained
popularity. These methods can bypass the need for constraints entirely.

\end{document}